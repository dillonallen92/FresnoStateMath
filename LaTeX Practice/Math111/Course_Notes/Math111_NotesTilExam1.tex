\documentclass[11pt]{amsart}
\usepackage{amssymb, amsmath, array, mathptmx, graphicx, multicol, footmisc, graphicx, comment, textcomp, bm, color}
\usepackage{amscd}
\usepackage{amsfonts}
\usepackage{soul,xcolor,mathtools}
\newcolumntype{C}{>$c<$}

\usepackage[sc,osf]{mathpazo}
\usepackage{tcolorbox}
\tcbuselibrary{theorems}

\newtcbtheorem[number within = section]{mytheo}{Theorem}%
{colback=gray!5,colframe=blue!35!black,fonttitle=\bfseries}{th}


	
%STICKERS
%\usepackage{eso-pic}
%\newcommand\AtPagemyUpperLeft[1]{\AtPageLowerLeft{
%\put(\LenToUnit{0.9\paperwidth},\LenToUnit{0.9\paperheight}){#1}}}
%\AddToShipoutPictureFG{
%  \AtPagemyUpperLeft{{\includegraphics[width=.5in,keepaspectratio]{Expert.jpg}}}
%}

\newtheorem{definition}{Definition}
\newtheorem{example}{Example}
\newtheorem{theorem}{Theorem}
\newtheorem{exercise}{Exercise}
\newtheorem{lemma}{Lemmma}

\usepackage[colorlinks = true,
            linkcolor = blue,
            urlcolor  = blue,
            citecolor = blue,
            anchorcolor = blue]{hyperref}


\setcounter{MaxMatrixCols}{30}
\setlength{\topmargin}{-.5in}
\setlength{\textheight}{8.75in}
\setlength{\oddsidemargin}{-0.2in}
\setlength{\evensidemargin}{-0.2in}
\setlength{\textwidth}{6.5in}
\def\labelenumi{\arabic{enumi}.}
\def\theenumi{\arabic{enumi}}
\def\labelenumii{(\alph{enumii})}
\def\theenumii{\alph{enumii}}
\def\p@enumii{\theenumi.}
\def\labelenumiii{\arabic{enumiii}.}
\def\theenumiii{\arabic{enumiii}}
\def\p@enumiii{(\theenumi)(\theenumii)}
\def\labelenumiv{\arabic{enumiv}.}
\def\theenumiv{\arabic{enumiv}}
\def\p@enumiv{\p@enumiii.\theenumiii}
\newcommand{\ve}{\varepsilon}
\newcommand{\C}{\mathbb{C}}
\newcommand{\R}{\mathbb{R}}
\newcommand{\N}{\mathbb{N}}
\newcommand{\Q}{\mathbb{Q}}
\newcommand{\Z}{\mathbb{Z}}
\newcommand{\ds}[1]{\displaystyle{ #1}}
\newcommand{\tr}{\textrm{tr}}
\newcommand{\seq}[1]{(#1_n)_{n=1}^{\infty}}
\newcommand{\ser}[1]{\sum_{n=1}^{\infty} #1_n}
\renewcommand{\vec}[1]{\mathbf{#1}}

\newcommand{\normal}{\trianglelefteq}
\newcommand{\mbf}{\mathbf}
\newcommand{\mbb}[1]{\mbox{\boldmath$#1$}}
\newcommand{\con}{\, | \,}
\newcommand{\ip}[2]{\left\langle #1, #2 \right\rangle}
\newcommand{\der}[2]{\ds{\frac{\partial #1}{\partial#2}}}
\newcommand{\se}[1]{\setlength{\extrarowheight}{#1}}
\newcommand{\G}[2]{\ds{\Gamma_{#1}^{#2}}}
\newcommand{\wtilde}{\widetilde}
\newcommand{\surface}{$\vec{r} : \mathcal{U} \to \mathbb{R}^3$}
\newcommand{\WTS}{\xRightarrow[]{\text{WTS}}}

\pagestyle{plain}
\setcounter{secnumdepth}{0}
\parindent=0pt

\newcommand{\oscar}[1]{\textcolor{red}{\textbf{#1}}}
\setstcolor{red}
%\st{Some overstruck text}

\begin{document}

\begin{center}
\textbf{MATH 111, Course Notes Until Exam 1}%CHANGE THIS EVERY WEEK
\end{center}

\vspace{.3 in}

\textbf{Name:}\underline{Dillon Allen} \hspace{1in} \textbf{Date:} \underline{08/30/2021}%CHANGE THIS EVERY WEEK

\vspace{.3in}

%%%%%%%%%%%%%%%%%%%%%%%%%%%%%%%%%%%
% 		  Start of Code           %
%%%%%%%%%%%%%%%%%%%%%%%%%%%%%%%%%%%


\begin{section}{Lecture 1}

In this lecture, we just learned the basics of LaTeX so nothing really happened.

\end{section}

\begin{section}{Lecture 2: Statements and Hypotheses}

\begin{enumerate}
	\item Statements 
	\begin{definition} 
	A statement is a sentence that is either \textbf{true} or \textbf{false}.
	\end{definition}
	
	\begin{example}
		The sky will is blue today.
	\end{example}
	
	\item Definition of a Proof and Theorem
	
	\begin{definition}
		A \textbf{proof} is an argument that is logically sound. This argument is  \textbf{thorough}, \textbf{logical}, and \textbf{complete}.
	\end{definition}
	
	\begin{definition}
		A \textbf{theorem} is a true statement that has a proof.
	\end{definition}
	
Theorem structure: If \underline{HYPOTHESIS} then \underline{CONCLUSION}.
	
This will later be written as: IF $ \underline{P} \implies \underline{Q}$ 
	
\end{enumerate}
	
\end{section}

\begin{section}{Lecture 3: Implications and Mathematical Writing}
This lecture was full of examples for writing mathematical statements into the $ P \implies Q$ format.

\begin{example}
	If $f$ is differentiable at $x = c$ then $f$ is continuous at $x = c$.
	
	\vspace{.1 in}
	
	$P = f \ \text{is differentiable at} x = c $ 
	
	\vspace{.1 in}
	
    $Q = f \ \text{is continuous at} x = c. $
\end{example}

\begin{example}
	$x^2 \geq 0, \ \text{for all} x \in \R $
	
	\vspace{.1in}
	
	$ P = x \in \R $
	
	\vspace{.1 in}
	
	$Q = x^2 \geq 0 $
	
\end{example}

\begin{example}
	If $x \in A \implies x \in B \ \text{and} \ B \subseteq C \ \text{then} \ x \in A \implies x \in C$.
	
\vspace{.1 in}

$ P = x \in A \implies x \in B \ \text{and} \ B \subseteq C $

\vspace{.1 in}

$ Q = x \in A \implies x \in C $
\end{example}	
\end{section}

\begin{section}{Lecture 4: AND, OR, Truth Tables}
	AND and OR Statements: The symbol for AND is $\wedge$ and the symbol for OR is $\vee$.
	
	Truth Tables
	\[
	\begin{array}{ C|C|C|C }
	$p$ & $q$ & $p \wedge q$ & $p \vee q$ \\
	\hline
	T & T & T & T\\
	T & F & F & T\\
	F & T & F & T\\
	F & F & F & F \\
	\end{array}
	\]
\end{section}

\begin{section}{Lecture 5: IFF and Negation}


\begin{definition} If $ P \implies Q$ and $Q \implies P$ are equivalent, then we can rewrite the expression as $ P \iff Q$ \end{definition}. This is known as the "If and Only If" expression.

\begin{definition}
	Reversing an implication is called the \textbf{converse}.
\end{definition}

\begin{definition}
	Given a statement $P$, the statement saying the exact opposite of $P$ is called the \textbf{negation} of $P$, denoted as $\neg P$
\end{definition}

\end{section}

\begin{section}{Lecture 6: DeMorgans Laws}

\begin{definition}
	DeMorgan's Laws are 
	
	\begin{enumerate}
		\item $ \neg \left(P \wedge Q \right) = \neg P \vee \neg Q $
		
		\item $ \neg \left( P \vee Q \right) = \neg P \wedge \neg Q $
	\end{enumerate}
\end{definition}

\begin{definition}
	$ P \implies Q$ is logically equivalent to $ \neg P \vee Q $
\end{definition}


Examples: Negate the following sentences

\begin{enumerate}
	\item It will be hot OR sunny tomorrow
	
	$ P = $ It will be hot tomorrow
	
	$ Q = $ It will be sunny tomorrow
	
	$ \neg \left( P \vee Q \right) = \neg P \wedge \neg Q $
	
	It will not be hot AND it will not be sunny tomorrow.
	
	\item All math professors are awesome
	
	This is tricky because it uses a quantifier $\forall $. $\neg \forall \implies \exists$ So the negation of this sentence is "Not all math professors are awesome" or "There exists a math professor that is not awesome".
\end{enumerate}
\end{section}

\begin{section}{Lecture 7: Quantifiers}

There are three quantifiers that we need to know

\begin{enumerate}
	\item \underline{For all}
	
	The first quantifier to know is the "for all" symbol, denoted as $\forall$. An example of this symbol is 
	
	\[
		\forall x \in \N, \ x^2 + 1 > 0
	\]
	
	\item \underline{There Exists}
	
	The next symbol dictates the existence of some values within a set or collection of objects. This is denoted by the symbol $\exists$. An example of this is 
	
	\[
		\exists x \in \Z, \ \sqrt{x} - 3 < 0
	\]
	
	\item \underline{Unique Existence}
	
	The final quantifier is a more specific version of existence. This one states that there is only one value, a unique value, inside a set or collection. This is denoted as $\exists !$. An example of this symbol is 
	
	\[
		\exists! n \in \N, \forall x \in \R, \ x^n = x
	\]
\end{enumerate}

\end{section}

\begin{section}{Negation of Quantifiers}


	There are two main takeaways from negating quantifiers: 
	
	\begin{enumerate}
		
		\item \underline{for all}
		
		The negation of for all, $\forall$, is $\neg \forall = \exists$.
		
		An example would be 
		
		\[
		\neg \forall x \in \N \implies \exists x \in \N
		\]
		
		\item \underline{There exists}
		
		The negation of there exists, $\exists$, is $\neg \exists = \forall $.
		
		An example of this negation would be 
		
		\[
			\neg ( \exists x \in \R, \ x^2 - 7x > 0)
		\]
		
		\[
			\implies \forall x \in \R, \ x^2 - 7x \leq 0
		\]
		
	\end{enumerate}
	
	A quick aside about inequalities. The negation of a strict inequality is the reverse, and includes said value. In the previous example, we had a strict inequality of being greater than 0, but the negation allowed us to be less than or equal to 0.
\end{section}



\end{document}