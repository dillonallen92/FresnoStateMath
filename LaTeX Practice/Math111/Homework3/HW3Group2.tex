\documentclass[11pt]{amsart}
\usepackage{amssymb, amsmath, mathptmx, graphicx, multicol, footmisc, graphicx, comment, textcomp, bm, color}
\usepackage{amscd}
\usepackage{amsfonts}
\usepackage{enumerate}

\usepackage[sc,osf]{mathpazo}

%STICKERS
%\usepackage{eso-pic}
%\newcommand\AtPagemyUpperLeft[1]{\AtPageLowerLeft{
%\put(\LenToUnit{0.9\paperwidth},\LenToUnit{0.9\paperheight}){#1}}}
%\AddToShipoutPictureFG{
%  \AtPagemyUpperLeft{{\includegraphics[width=.5in,keepaspectratio]{Expert.jpg}}}
%}

\newtheorem{definition}{Definition}
\newtheorem{example}{Example}
\newtheorem{theorem}{Theorem}
\newtheorem{lemma}{Lemma}

\setcounter{MaxMatrixCols}{30}
\setlength{\topmargin}{-.5in}
\setlength{\textheight}{8.5in}
\setlength{\oddsidemargin}{0.0in}
\setlength{\evensidemargin}{0.0in}
\setlength{\textwidth}{6.5in}
\def\labelenumi{\arabic{enumi}.}
\def\theenumi{\arabic{enumi}}
\def\labelenumii{(\alph{enumii})}
\def\theenumii{\alph{enumii}}
\def\p@enumii{\theenumi.}
\def\labelenumiii{\arabic{enumiii}.}
\def\theenumiii{\arabic{enumiii}}
\def\p@enumiii{(\theenumi)(\theenumii)}
\def\labelenumiv{\arabic{enumiv}.}
\def\theenumiv{\arabic{enumiv}}
\def\p@enumiv{\p@enumiii.\theenumiii}
\newcommand{\ve}{\varepsilon}
\newcommand{\C}{\mathbb{C}}
\newcommand{\R}{\mathbb{R}}
\newcommand{\N}{\mathbb{N}}
\newcommand{\Q}{\mathbb{Q}}
\newcommand{\Z}{\mathbb{Z}}
\newcommand{\ds}[1]{\displaystyle{ #1}}
\newcommand{\tr}{\textrm{tr}}
\newcommand{\seq}[1]{(#1_n)_{n=1}^{\infty}}
\newcommand{\ser}[1]{\sum_{n=1}^{\infty} #1_n}
\renewcommand{\vec}[1]{\mathbf{#1}}

\newcommand{\normal}{\trianglelefteq}
\newcommand{\mbf}{\mathbf}
\newcommand{\mbb}[1]{\mbox{\boldmath$#1$}}
\newcommand{\con}{\, | \,}
\newcommand{\ip}[2]{\left\langle #1, #2 \right\rangle}
\newcommand{\der}[2]{\ds{Frac{\partial #1}{\partial#2}}}
\newcommand{\se}[1]{\setlength{\extrarowheight}{#1}}
\newcommand{\G}[2]{\ds{\Gamma_{#1}^{#2}}}
\newcommand{\wtilde}{\widetilde}
\newcommand{\surface}{$\vec{r} : \mathcal{U} \to \mathbb{R}^3$}
\pagestyle{plain}
\setcounter{secnumdepth}{0}
\parindent=0pt

\newcommand{\oscar}[1]{\textcolor{red}{\textbf{#1}}}

\begin{document}
\begin{center}
\textbf{MATH 111, Assignment 3 (due 09/14/21)}%CHANGE DATE EVERY WEEK
\end{center}

\vspace{.1in}

\begin{center}
\textbf{MATH 111, Assignment 3 Solutions}%CHANGE THIS EVERY WEEK
\end{center}

\vspace{.1 in}

\textbf{Name:}\underline{\ Dillon Allen, Julio Hernandez } \hspace{1in} \textbf{Date:} \underline{ 09/11/2021}%CHANGE THIS EVERY WEEK

\textbf{Group Name:}\underline{\ Group 2}


\begin{center}
\begin{tabular}{|p{2.5in}|}
\hline

\hline
\end{tabular}
\end{center}

\vspace{.2in}

\begin{enumerate}
%%%%%%%%%%%%%%%%%%%%%%%%%%%%%%%%%%%%%%%%%%%%%%%%%%%%%%%%%%
\item (50 points) Negate the following statements. Give the statement \emph{and} its negation in both symbolic form, then also give negation in the form of a sentence. Finally, write the converse of each statement when possible.

\begin{enumerate}[(i)]
\item The numbers $x$ and $y$ are both odd.

Solution: The statement can be written in $P \wedge Q$ in symbolic form. Thus, the statement reads $(x\text{ is odd}) \wedge (y\text{ is odd})$. The negation of the symbolic form is

\[
\neg(x\text{ is odd}) \wedge (y\text{ is odd}) = (x\text{ is even}) \vee (y\text{ is even})
\]

The negation in word form can be described as, The number $x$ is even or the number $y$ is even.


\item Let $ABC$ be an equilateral triangle or a right triangle.

Solution: P: ABC is an equilateral triangle

Q: ABC is a right triangle.

This statement is of the form $P \vee Q$, so the negation of this is

\[
    \neg(P \vee Q) = \neg P \wedge \neg Q
\]

The negation in sentence form is that ABC is not an equilateral triangle and ABC is not a right triangle.


\item $\sqrt{2}$ is a real number that is not rational.

Solution: P: $\sqrt{2} \in \R$

Q: $\sqrt{2} \notin \Q$

Symbolically, this is an and statement that is of the form $P \wedge Q$. The negation of this is $\neg( P \wedge Q) = \neg P \vee \neg Q$. In a sentence, this reads as $\sqrt{2}$ is not a real number or $\sqrt{2}$ is a rational number.


\item If $x$ is a real number then $x^2+1>0.$ 

Solution: P: $x \in \R$

Q: $x^2 + 1 > 0$

Symbolically this is a $P \implies Q$ statement. The negation of this is is $\neg(P \implies Q)$ = $P \wedge \neg Q$. In a symbolic form the negation this reads as $(x \in \R) \wedge (x^2 +1 \leq 0) $. In a sentence the negation reads as $x$ is a real number and one added to the square of x is less than or equal to 0. The converse of this statement would be that if $x^2+1 > 0$ then $x$ is a real number.

\item The number $x$ is positive, but the number $y$ is not positive.

Solution:The original statement in symbolic form is $(x > 0) \wedge (y < 0)$. The negation of this statement is 
\begin{align*}
\neg ((x > 0) \wedge (y < 0)) &= \neg (x > 0) \vee \neg (y < 0) \\
                              &= (x \leq 0) \vee (y \geq 0) \\
\end{align*}

In sentence form, this statement says that the number $x$ is less than or equal to zero or the number $y$ is greater than or equal to zero.

\item If $x$ is a rational number and $x\neq 0$, then $\tan(x)$ is not a rational number.

Solution: The original statement in symbolic form would be $$(x \in \Q) \wedge (x \neq 0) \implies (\tan(x) \notin \Q)$$. The negation of the following statement would be 
\begin{align*}
    \neg((P \wedge Q)&\implies R)\\
    (P \wedge Q) & \wedge \neg R\\
\end{align*}
% ~[( p AND q) -> R]
% (p AND q) AND ~R


Symbolically, the negation says 

\[
    ((x \in \Q) \wedge (x \neq 0)) \wedge ( \tan(x) \in \Q)
\]

In a sentence, this negation says that $x$ is a rational number and $ x \neq 0 $ and $\tan(x)$ is a rational number. 

For the converse, the statement would be $ R \implies ( P \wedge Q) $. In sentence form, this would say if $\tan(x)$ is not a rational number then $x$ is a rational number and $x \neq 0$.

\item For all real numbers $x$ and $y$, $x\neq y$ implies that $x^2+y^2>0$.

Solution: Symbolically, this statement reads
\[ \forall x \in \R \ \forall y \in \R, \ (x \neq y) \implies (x^2 + y^2 > 0) \]

This is a $ P \implies Q$ statement, with the negation being $\neg (P \implies Q) = P \wedge \neg Q$. Therefore, the negation of this statement is

\begin{align*}
    \neg (\forall x \in \R \ \forall y \in \R, \ (x \neq y) \implies (x^2 + y^2 > 0)) &= \exists x \in \R \ \exists y \in \R, \ \neg ( (x \neq y) \implies (x^2 + y^2 > 0)) \\
    &= \exists x \in \R \ \exists y \in \R, \ (x \neq y) \wedge \neg(x^2 + y^2 > 0) \\
    &= \exists x \in \R \ \exists y \in \R, \ (x \neq y) \wedge (x^2 + y^2 \leq 0)
\end{align*}

The negation, in sentence form,  says that there exists some real numbers $x$ and $y$ such that if $x \neq y$ then $x^2 + y^2 \leq 0$.

The converse of this statement says that $x^2 + y^2 > 0$ implies that $x$ and $y$ are real numbers where $x \neq y$.

\item There exists a rational number whose square is 2.

Solution Symbolically this statement read as 
\[ 
\exists x \in Q, \ x^2 = 2.
\]
This is a $P$ statement with the negation being $\neg P$
\begin{align*}
    \neg(\exists x \in Q, x^2 &= 2)\\
    (\forall x \in Q, x^2 & \neq 2)\\
\end{align*}
This negation in as a sentence means says that for all $x \in \Q$ there is no $Q$ that's $x^2 \neq 2$


\item If the set $B$ is contained in the set $A$, then $B$ without $A$ is nonempty.

Solution: 

P: The set $B$ is contained in set $A$.

Q: $B$ without $A$ is nonempty. 

$P \implies Q$ statement. Therefore, the negation of this would be 

\[
    \neg(P \implies Q) = P \wedge \neg Q
\]

In a sentence, this negation reads as: The set $B$ is contained in set $A$ and $B$ without $A$ is empty.

The converse of this sentence is if set $B$ without $A$ is nonempty, then the set $B$ is contained in the set $A$.

\item For every prime number $p$, either $p$ is odd or $p$ is 2.

Solution: Symbolically, the statement can be written as 

\[ 
    \text{for all primes } p, \ (p \ \text{odd}) \vee ( p = 2)
\]

The statement is of the form $\forall x, P \vee Q $ so the general negation is $ \exists x, \neg P \wedge \neg Q$. So, the negation of this statement is 

\[
    \text{There exists some prime } p, \ \neg (p \ \text{odd}) \wedge \neg (p = 2)
\]
    
\[
    \text{There exists some prime } p, \text{ such that }, \  (p \ \text{even}) \wedge (p \neq 2)
\]

In sentence form, there exists a prime $p$ such that $p$ is even and $p$ is not equal to $2$.


\end{enumerate}

\vspace{.2in}

%%%%%%%%%%%%%%%%%%%%%%%%%%%%%%%%%%%%%%%%%%%%%%%%%%%%%%%%%%%%%%%%%%
\item (30 points) Write each of the following as English sentences. Say whether they are true or false. Prove your claim (this may need examples and/or counterexamples)
\begin{enumerate}[(i)]
\item $(\forall x\in\mathbb{N})(x \ge 1).$

Solution: For all natural numbers $x$, such that $x$ is greater than or equal to $1$

We know this to be true as natural numbers are defined as $\N = \lbrace 1,2,3,\dots \rbrace$. Therefore if we were to take the number $2$ as an example, we get $2 \ge 1$. Which we know satisfies our condition. The lowest number we could use would be $1$ and if we plug that in we get $1 \ge 1$. By this we know that all natural numbers will satisfy our condition. 

\item $(\exists! x\in\mathbb{R})(x\ge0\wedge x\leq0).$

Solution: There is exactly one number such that x is greater than or equal to $0$ and also less than or equal to $0$. This is a true statement for only the number $0$. 

\begin{proof}

We will prove this by taking 3 cases, for $x > 0$, $x = 0$, and $x < 0$.


Case 1: Taking $x > 0$, let $x = 5$. $5 \geq 0$, which is true, but $5 \leq 0$ is false. Therefore, the logical statement $P \wedge Q$ is false.

Case 2: Let $x = 0 $. $0 \geq 0$ is true and $0 \leq 0$ is also true. So, logically, $T \wedge T = T$. This is a true statement.

Case 3: Taking $x < 0$, let $x = -2$. $ -2 \geq 0$ is false, whereas $-2 \leq 0$ is true. $F \wedge T = F$, therefore this case is false.

Therefore, the only case that satisfies this claim is when $x = 0$. 
\end{proof}



\item $\forall x\in\mathbb{R},\,\exists y\in\mathbb{R},\,x+y=0$.

Solution:For all real numbers $x$, there is some number $y$ such that their sum is zero. This is a true statement. For every value x, we have the equation $x = -y$. So no matter what value of $x$ we have, there will exist a $y$ value that is the additive inverse of $x$.


\item $\exists x\in\mathbb{R},\,\forall y\in\mathbb{R},\,x+y=0$.

Solution: For exists some real number x such that for all real numbers y, their sum is zero. This statement is false, because you can not have a value for $x$ that is the additive inverse for all numbers $y$. 


\item $\exists x\in\mathbb{R},\,\forall y\in\mathbb{R},\,x\geq y$.

Solution: There is a real number $x$ that is bigger than all real numbers $y$. This is a true statement, which has to do with the infinities of the real numbers. I can always find some number bigger than the previous ones suggested. Whenever you can guess a number $x$, I can always guess $10^x$. This argument can continue building until we exhaust all numbers, but will come up with new real numbers in the process.


\end{enumerate}


\vspace{.2in}

%%%%%%%%%%%%%%%%%%%%%%%%%%%%%%%%%%%%%%%%%%%%%%%%%%%%%%%%%%%%%%%%%%
\item (20 points) Translate each of the following sentences into symbolic logic. 
\begin{enumerate}[(i)]
\item Even numbers are divisible by 2.

Solution: Symbolically, this can be rewritten into a $ P \implies Q $ conditional with:

P: $n$ is an even number.

Q: $n$ is divisible by 2.

Therefore, we have $ P \implies Q $ rewriting the statement as "if $n$ is an even number then it is divisible by 2".


\item If $f$ is a polynomial, $f'$ is constant whenever the degree of $f$ is less than 2.

Solution: Let $p(x)$ represent all polynomial functions and deg(f) represent the degree of some function f. Then $$(\forall f \in p(x)), \ (\text{deg}(f) < 2) \implies (f' = \text{constant})$$


\item If $x$ is prime, then $\sqrt{x}$ is not a rational number.

Solution: $x \in \lbrace \text{primes} \rbrace \implies \sqrt{x} \notin \Q$


\item Every set is contained in itself.

Solution: P: A is a set.

Q: A is a subset of itself.

$P \implies Q$ says that if A is a set then A is a subset of itself.


\item There exist two integers whose product is negative, but whose sum is positive.

Solution: 

This is a $ P \wedge Q $ statement.

$(\exists x \in \Z,  \exists y \in \Z)$, $(x*y < 0) \wedge (x + y > 0)$


\end{enumerate}
\end{enumerate}


\underline{\hspace{6in}}
%%%%%%%%%%%%%%%%%%%%%%%%%%%%%%%%%%%%%%%%%%%%%%%%%%%%%%%%
\vspace{.5in}

\noindent \textbf{Extra Credit:}  Write each of the following as English sentences. Say whether they are true or false. Prove your claim (this may need examples and/or counterexamples)
\begin{enumerate}[(i)]
\item $(\forall x\in\mathbb{N})(x \text{ is prime}\wedge x\ne 2\implies x \text{ is odd}).$
\item $(\exists! x\in\mathbb{R})( \log_e{x} = 1).$ 
\item $\forall x\in\mathbb{R},\,x+2>x$.
\item $\exists x\in\mathbb{N},\, 2x+3\geq6x+7$.
\end{enumerate}




\end{document}