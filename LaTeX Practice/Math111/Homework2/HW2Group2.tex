\documentclass[11pt]{amsart}
\usepackage{amssymb, amsmath, mathptmx, graphicx, multicol, footmisc, graphicx, comment, textcomp, bm, color}
\usepackage{amscd}
\usepackage{amsfonts}

\usepackage[sc,osf]{mathpazo}

%STICKERS
%\usepackage{eso-pic}
%\newcommand\AtPagemyUpperLeft[1]{\AtPageLowerLeft{
%\put(\LenToUnit{0.9\paperwidth},\LenToUnit{0.9\paperheight}){#1}}}
%\AddToShipoutPictureFG{
%  \AtPagemyUpperLeft{{\includegraphics[width=.5in,keepaspectratio]{Expert.jpg}}}
%}

\newtheorem{definition}{Definition}
\newtheorem{example}{Example}
\newtheorem{theorem}{Theorem}
\newtheorem{lemma}{Lemma}

\setcounter{MaxMatrixCols}{30}
\setlength{\topmargin}{-.5in}
\setlength{\textheight}{8.5in}
\setlength{\oddsidemargin}{0.0in}
\setlength{\evensidemargin}{0.0in}
\setlength{\textwidth}{6.5in}
\def\labelenumi{\arabic{enumi}.}
\def\theenumi{\arabic{enumi}}
\def\labelenumii{(\alph{enumii})}
\def\theenumii{\alph{enumii}}
\def\p@enumii{\theenumi.}
\def\labelenumiii{\arabic{enumiii}.}
\def\theenumiii{\arabic{enumiii}}
\def\p@enumiii{(\theenumi)(\theenumii)}
\def\labelenumiv{\arabic{enumiv}.}
\def\theenumiv{\arabic{enumiv}}
\def\p@enumiv{\p@enumiii.\theenumiii}
\newcommand{\ve}{\varepsilon}
\newcommand{\C}{\mathbb{C}}
\newcommand{\R}{\mathbb{R}}
\newcommand{\N}{\mathbb{N}}
\newcommand{\Q}{\mathbb{Q}}
\newcommand{\Z}{\mathbb{Z}}
\newcommand{\ds}[1]{\displaystyle{ #1}}
\newcommand{\tr}{\textrm{tr}}
\newcommand{\seq}[1]{(#1_n)_{n=1}^{\infty}}
\newcommand{\ser}[1]{\sum_{n=1}^{\infty} #1_n}
\renewcommand{\vec}[1]{\mathbf{#1}}

\newcommand{\normal}{\trianglelefteq}
\newcommand{\mbf}{\mathbf}
\newcommand{\mbb}[1]{\mbox{\boldmath$#1$}}
\newcommand{\con}{\, | \,}
\newcommand{\ip}[2]{\left\langle #1, #2 \right\rangle}
\newcommand{\der}[2]{\ds{Frac{\partial #1}{\partial#2}}}
\newcommand{\se}[1]{\setlength{\extrarowheight}{#1}}
\newcommand{\G}[2]{\ds{\Gamma_{#1}^{#2}}}
\newcommand{\wtilde}{\widetilde}
\newcommand{\surface}{$\vec{r} : \mathcal{U} \to \mathbb{R}^3$}
\pagestyle{plain}
\setcounter{secnumdepth}{0}
\parindent=0pt

\newcommand{\oscar}[1]{\textcolor{red}{\textbf{#1}}}

\begin{document}
\begin{center}
\textbf{MATH 111, Assignment 2 (due 09/07/21)}%CHANGE DATE EVERY WEEK
\end{center}

\vspace{.1in}

\begin{center}
\textbf{MATH 111, Assignment 2 Solutions}%CHANGE THIS EVERY WEEK
\end{center}

\vspace{.1 in}

\textbf{Name:}\underline{\ Dillon Allen, Jeffrey Seeto, Julio Hernandez } \hspace{1in} \textbf{Date:} \underline{ 09/03/2021}%CHANGE THIS EVERY WEEK

\textbf{Group Name:}\underline{\ Group 2}


\begin{center}
\begin{tabular}{|p{2.5in}|}
\hline

\hline
\end{tabular}
\end{center}

\vspace{.2in}

\begin{enumerate}
%%%%%%%%%%%%%%%%%%%%%%%%%%%%%%%%%%%%%%%%%%%%%%%%%%%%%%%%%%
\item (21 points) Determine whether or not the following are statements. In the case of a statement, say if it is true or false. 
\begin{enumerate}
\item Multiply $2x+2$ by $2$.

Solution: This is not a statement, this is a command or a step in doing some sort of calculation. Because of this, there is nothing to verify for true or false values. Now, if there was more context, or a claim that this step leads to something else, then we could have a statement that is true or false.

\item Is $3\times 6=15$?

Solution: This is not a statement, this is a question.

\item The integer $0$ is neither even nor odd.

Solution: This is a statement, but it is false. The definition of an even number $n = 2k, \ \text{where} \ k \in \Z$. So for $ k =0 $, we can get a value $n$. Therefore, $0$ is an even number.

\item If $p(x)$ is a polynomial of degree $5$, then $p'(x)$ is a polynomial of degree $4$.

Solution: This is a statement and it is true. Since the derivative of a polynomial $p(x)$ of degree $n$ is a polynomial of degree $n-1$ . With n being the degree of the polynomial. Thus, since $n = 5, \ n-1 = 4$.

\item Set $\mathbb{Q}$.

Solution:This is not a statement as nothing was claimed or stated that has a true or false value to it.

\item For $x$ a real number, $\sqrt{x^2}=x$.

Solution: This is a statement but it is false. When taking the square root of a real number, $x$, we get $\sqrt{x^2} = \pm x$. So. this is only half of the solutions for the given statement.

\item $x(x+1)=4$

Solution: This is not a statement, this is an equation with no context on any values for $x$. Due to no context, there is no way to verify if this is strictly true or false.
\end{enumerate}

%%%%%%%%%%%%%%%%%%%%%%%%%%%%%%%%%%%%%%%%%%%%%%%%%%%%%%%%
\vspace{.2in}

\item (21 points) Without changing their meanings, convert each of the following sentences into a sentence having the form $P\implies Q$, $P\wedge Q$, $P\vee Q$, $\neg P$, or $P\iff Q$. Be sure to state exactly what statements $P$ and $Q$ stand for. \\
Provide the converse of the statement when possible. 
\begin{enumerate}
\item $\sqrt{3}\not\in\mathbb{Q}$

Solution: This will be best written as a negation of P statement, with P : $\sqrt{3} \in \Q$. Therefore, the negation of P is $\neg P : \sqrt{3} \not\in  \Q$

\item The differentiability of $f$ is a sufficient condition for $f$ to be continuous.

Solution: This will be best written as $P\implies Q$ statement. 

P: $f$ is differentiable

Q: $f$ is continuous. 

The converse of this statement would be: If $f$ is continuous it is a sufficient condition that $f$ is then differentiable. 




\item $n$ is a multiple of 5 only if $n$ is a multiple of 25.

Solution: This would be best written as an if, then statement, $P \implies Q$. 

P : $n$ is a multiple of 25. 

Q: $n$ is a multiple of 5. 

The converse of this statement would be $Q \implies P$. Which states that if n is a multiple of 5 then it is a multiple of 25.

\item If a function has a constant derivative then it is linear, and conversely. 

Solution: This would best be written as an if and only if, $P\iff Q$, Where

P = A function has a constant derivative

Q = A function is linear. 

The converse of this statement, $Q\iff P$, states that if a function is linear then it has a constant derivative.

\item Whenever a triangle is acute, one interior angle is at least $90^\circ$.

Solution: This would be best written as an if then statement, $P\implies Q$. Where 

P = A triangle is acute,

Q = it has one interior angle that is at least $90^\circ$. 

The converse of this statement, $Q\implies P$, states that if there is at least a $90^\circ$ interior angle the triangle is acute.

\item If a function is continuous, then it is differentiable

Solution: This will be a $P \implies Q$ statement, where 

$P =$ a function is continuous 

$Q = $ a function is differentiable. 

Therefore, $ P \implies Q$. The converse of this statement, $Q \implies P$, states that if the function $f$ is differentiable, then it is continuous.



\item A function $p(x)$ being a polynomial is a necessary and sufficient condition for the derivative $p'(x)$ to be a polynomial.


Solution: This is a $ P \iff Q $ statement with

P = The derivative $p^{'}(x)$ is a polynomial

Q = The function $p(x)$ is a polynomial

The converse for this statement is $ Q \iff P $, which says if $p(x)$ is a polynomial then the derivative $p^{'}(x)$ will be a polynomial. This sounds the same for both sides because of the nature of an "If-and-only-If" statement.

\end{enumerate}

%%%%%%%%%%%%%%%%%%%%%%%%%%%%%%%%%%%%%%%%%%%%%%%%%%%%%%%%
\vspace{.2in}

\item (20 points) Suppose the statement $((P\vee S)\wedge R)\implies (R\wedge S)$ is false. Find the truth values of $P,$ $Q$, $R$, and $S$. Justify each step of your reasoning.


Solution: Since the overall statement is of the form $P \implies Q$ and has a truth value of False, that means we are looking for when $P$ is true and $Q$ is false, symbolically as $ T \implies F$. Therefore, we are looking for the truth value of $((P \vee S) \wedge R)$ to be True and the truth value of $(R \wedge S) $ to be false. If $( R \wedge S) $ is false, that means either R is false or S is false. Looking at the left hand side, $(P \vee S) \wedge R$ is true, which means R must be true and $(P \vee S) $ must also be true, due to the AND statement needing truth values (T AND T) to be true. $(P \vee S)$ has three different scenarios that lead to a truth value of true: P = T S = T, P = T S = F, P = F S = T. Since R = True and $(R \wedge S) $ is false, that leads to S = False. That leaves us with P = True.

In conclusion: P = True, S = False, R = True.


%%%%%%%%%%%%%%%%%%%%%%%%%%%%%%%%%%%%%%%%%%%%%%%%%%%%%%%%
\vspace{.2in}

\item (20 points) Negate the following sentences. Give the statement \emph{and} its negation in both symbolic form, then also give negation in the form of a sentence.
\begin{enumerate}
\item The number $x$ is positive, but the number $y$ is not positive.

Solution: The original statement in symbolic form is $(x > 0) \wedge (y < 0)$. The negation of this statement is 
\begin{align*}
\neg ((x > 0) \wedge (y < 0)) &= \neg (x > 0) \vee \neg (y < 0) \\
                              &= (x \leq 0) \vee (y \geq 0) \\
\end{align*}

In sentence form, this statement says that the number $x$ is negative or the number $y$ is positive.

\item If $x$ is a rational number and $x \neq 0$, then $\tan(x)$ is not a rational number.


Solution: The original statement in symbolic form would be $x \in \Q$ and $x\neq 0$ then $\tan(x) \neq \Q$. The negation of the following statement would be 
\begin{align*}
    \neg(P \wedge Q)&\implies R\\
    (P \wedge Q) & \wedge \neg R\\
\end{align*}
% ~[( p AND q) -> R]
% (p AND q) AND ~R


Symbolically, the negation says 

\[
    ((x \in \Q) \wedge (x \neq 0)) \wedge ( \tan(x) \in \Q)
\]

In a sentence, this negation says that $x$ is a rational number and $ x \neq 0 $ and $\tan(x)$ is a rational number. 

%%%%%%%%%%% TODO: THIS SOLUTION NEEDS TO BE CHANGED %%%%%%%%%%%%%%%%%%%%%%

\item For all real numbers $x$ and $y$, $x\neq y$ implies that $x^2+y^2>0$.

Solution: Symbolically, this statement reads
\[ \forall x \in \R \ \forall y \in \R, \ (x \neq y) \implies (x^2 + y^2 > 0) \]

This is a $ P \implies Q$ statement, with the negation being $\neg (P \implies Q) = P \wedge \neg Q$. Therefore, the negation of this statement is

\begin{align*}
    \neg (\forall x \in \R \ \forall y \in \R, \ (x \neq y) \implies (x^2 + y^2 > 0)) &= \exists x \in \R \ \exists y \in \R, \ \neg ( (x \neq y) \implies (x^2 + y^2 > 0)) \\
    &= \exists x \in \R \ \exists y \in \R, \ (x \neq y) \wedge \neg(x^2 + y^2 > 0) \\
    &= \exists x \in \R \ \exists y \in \R, \ (x \neq y) \wedge (x^2 + y^2 \leq 0)
\end{align*}

The negation, in sentence form,  says that there exists some real numbers $x$ and $y$ such that if $x \neq y$ then $x^2 + y^2 \leq 0$.

\item There exists a rational number whose square is 2.
% exists x in Q, x^2 = 2.
% for all x in Q, x^2 != 2

Solution: Symbolically this statement read as 
\[ 
\exists x \in \Q, \ x^2 = 2.
\]
This is a $P$ statement with the negation being $\neg P$
\begin{align*}
    \neg(\exists x \in \Q, \ x^2 &= 2)\\
    (\forall x \in \Q, \ x^2 & \neq 2)\\
\end{align*}
This negation in as a sentence means says that for all rational numbers $x$, $x \in \Q$,  there is no such $x$ where $x^2 = 2$.


%%%%%%%%%%%%%% TODO: THIS SOLUTION NEEDS TO BE MADE SIMPLER %%%%%%%%%%%%%%%%%%%

\item For every prime number $p$, either $p$ is odd or $p$ is $2$.

Solution: Symbolically, the statement can be written as 

\[ 
    \text{for all primes } p, \ (p \ \text{odd}) \vee ( p = 2)
\]

The statement is of the form $\forall x, P \vee Q $ so the general negation is $ \exists x, \neg P \wedge \neg Q$. So, the negation of this statement is 

\[
    \text{There exists some prime } p, \ \neg (p \ \text{odd}) \wedge \neg (p = 2)
\]
    
\[
    \text{There exists some prime } p, \text{ such that }, \  (p \ \text{even}) \wedge (p \neq 2)
\]

In sentence form, there exists a prime $p$ such that $p$ is even and $p$ is not equal to $2$.
\end{enumerate}

\vspace{.2in}

%%%%%%%%%%%%%%%%%%%%%%%%%%%%%%%%%%%%%%%%%%%%%%%%%%%%%%%%%%
\item (18 points) Typeset the following text using \LaTeX. \\

From the left table, it appears that we can make $f$ as close as we want to 3 by taking x sufficiently close to $-1$, which suggests that $\lim\limits_{x\to1}f(x) = 3$. This is also consistent with the graph of $f$. To see this a bit more rigorously and from an algebraic point of view, consider the formula for $f$:
$$f(x) = \frac{4-x^2}{x+2} .$$
The numerator and denominator are each polynomial functions, which are among the most well-behaved functions that exist. Formally, such functions are  \emph{continuous}, which means that the limit of the function at any point is equal to its function value. Here, it follows that as $x \rightarrow -1$, $(4-x^2) \rightarrow (4-(-1)^2) = 3$, and $(x+2) \rightarrow (-1+2) = 1$, so as $x \rightarrow -1$, the numerator of $f$ tends to 3 and the denominator tends to 1, hence $$\lim_{x \to -1}f(x) = \frac{3}{1} = 3$$
The situation is more complicated when $x \rightarrow -2$, due in part to the fact that $f(-2)$ is not defined. If we attempt to use a similar algebraic argument regarding the numerator and denominator, we observe that as $x \rightarrow -2$, $(4-x^2) \rightarrow (4-(-2)^2) = 0$, and $(x+2) \rightarrow (-2+2) = 0$, so as $x\rightarrow -2$, the numerator of $f$ tends to 0 and the denominator tends to 0. We call $0/0$ an \emph{indeterminate form} and will revisit several important issues surrounding such quantities later in the course. For now, we simply observe that this tells us there is somehow more work to do. From the table and the graph, it appears that $f$ should have a limit of $4$ at $x = -2$. To see algebraically why this is the case, let's work directly with the form of $f(x)$. Observe that \begin{align*} \lim_{x\to-2}f(x) & = \lim_{x\to-2} \frac{4-x^2}{x+2}\\
&=\lim_{x\to-2} \frac{(2-x)(2+x)}{x+2}.\\
&=\lim_{x\to-2} 2-x.\\
\end{align*}
Note that the last step is possible to the fact that since we are taking the limit as $x \rightarrow -2$, we are considering x values that are close, but not equal, to $-2$. Since we never actually allow $x$ to equal $-2$, the quotient $$\frac{2+x}{x+2}$$
has value $1$ for every possible value of $x$.

\noindent Because $2-x$ is simply a linear function, this limit is now easy to determine, and its value clearly is $4$. Thus, from several points of view we've seen that $$\lim_{x\to-2}f(x) = 4$$







\end{enumerate}


\underline{\hspace{6in}}
%%%%%%%%%%%%%%%%%%%%%%%%%%%%%%%%%%%%%%%%%%%%%%%%%%%%%%%%
\vspace{.5in}

\noindent \textbf{Extra Credit:} Without changing their meanings, convert each of the following sentences into a sentence having the form $P\implies Q$, $P\wedge Q$, $P\vee Q$, $\neg P$, or $P\iff Q$. Be sure to state exactly what statements $P$ and $Q$ stand for. \\
Provide the converse of the statement when possible. 
\begin{enumerate}
\item Triangles have four sides if squares have three sides.\\
\emph{Is this statement true or false? Why?}

\item A sequence $\{a_n\}_{n=1}^\infty$ is bounded whenever $\{a_n\}_{n=1}^\infty$ is convergent.\\
\emph{Is this statement true or false for all sequences? Why?}

\item Exactly one of the real numbers $x$ or $y$ is negative is equivalent to the product $xy$ being negative.\\
\emph{Is this statement true or false for all real numbers $x,y$? Why?}
\end{enumerate}
\end{document}