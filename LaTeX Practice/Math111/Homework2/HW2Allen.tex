\documentclass[11pt]{amsart}
\usepackage{amssymb, amsmath, mathptmx, graphicx, multicol, footmisc, graphicx, comment, textcomp, bm, color}
\usepackage{amscd}
\usepackage{amsfonts}
\usepackage{soul,xcolor,mathtools}

\usepackage[sc,osf]{mathpazo}

%STICKERS
%\usepackage{eso-pic}
%\newcommand\AtPagemyUpperLeft[1]{\AtPageLowerLeft{
%\put(\LenToUnit{0.9\paperwidth},\LenToUnit{0.9\paperheight}){#1}}}
%\AddToShipoutPictureFG{
%  \AtPagemyUpperLeft{{\includegraphics[width=.5in,keepaspectratio]{Expert.jpg}}}
%}

\newtheorem{definition}{Definition}
\newtheorem{example}{Example}
\newtheorem{theorem}{Theorem}
\newtheorem{exercise}{Exercise}
\newtheorem{lemma}{Lemmma}

\usepackage[colorlinks = true,
            linkcolor = blue,
            urlcolor  = blue,
            citecolor = blue,
            anchorcolor = blue]{hyperref}


\setcounter{MaxMatrixCols}{30}
\setlength{\topmargin}{-.5in}
\setlength{\textheight}{8.75in}
\setlength{\oddsidemargin}{-0.2in}
\setlength{\evensidemargin}{-0.2in}
\setlength{\textwidth}{6.5in}
\def\labelenumi{\arabic{enumi}.}
\def\theenumi{\arabic{enumi}}
\def\labelenumii{(\alph{enumii})}
\def\theenumii{\alph{enumii}}
\def\p@enumii{\theenumi.}
\def\labelenumiii{\arabic{enumiii}.}
\def\theenumiii{\arabic{enumiii}}
\def\p@enumiii{(\theenumi)(\theenumii)}
\def\labelenumiv{\arabic{enumiv}.}
\def\theenumiv{\arabic{enumiv}}
\def\p@enumiv{\p@enumiii.\theenumiii}
\newcommand{\ve}{\varepsilon}
\newcommand{\C}{\mathbb{C}}
\newcommand{\R}{\mathbb{R}}
\newcommand{\N}{\mathbb{N}}
\newcommand{\Q}{\mathbb{Q}}
\newcommand{\Z}{\mathbb{Z}}
\newcommand{\ds}[1]{\displaystyle{ #1}}
\newcommand{\tr}{\textrm{tr}}
\newcommand{\seq}[1]{(#1_n)_{n=1}^{\infty}}
\newcommand{\ser}[1]{\sum_{n=1}^{\infty} #1_n}
\renewcommand{\vec}[1]{\mathbf{#1}}

\newcommand{\normal}{\trianglelefteq}
\newcommand{\mbf}{\mathbf}
\newcommand{\mbb}[1]{\mbox{\boldmath$#1$}}
\newcommand{\con}{\, | \,}
\newcommand{\ip}[2]{\left\langle #1, #2 \right\rangle}
\newcommand{\der}[2]{\ds{\frac{\partial #1}{\partial#2}}}
\newcommand{\se}[1]{\setlength{\extrarowheight}{#1}}
\newcommand{\G}[2]{\ds{\Gamma_{#1}^{#2}}}
\newcommand{\wtilde}{\widetilde}
\newcommand{\surface}{$\vec{r} : \mathcal{U} \to \mathbb{R}^3$}
\newcommand{\WTS}{\xRightarrow[]{\text{WTS}}}

\pagestyle{plain}
\setcounter{secnumdepth}{0}
\parindent=0pt

\newcommand{\oscar}[1]{\textcolor{red}{\textbf{#1}}}
\setstcolor{red}
%\st{Some overstruck text}

\begin{document}

\begin{center}
\textbf{MATH 111, Assignment 2 Solutions}%CHANGE THIS EVERY WEEK
\end{center}

\vspace{.3 in}

\textbf{Name:}\underline{\ Dillon Allen} \hspace{1in} \textbf{Date:} \underline{09/01/2021}%CHANGE THIS EVERY WEEK

\textbf{Group Name:}\underline{\ Group 2 (TBA)}

\vspace{.3in}

\begin{enumerate}

\item Determine whether or not the following are staements. In the case of a statement, say if it is true or false.

\begin{enumerate}
	\item Multiply $2x + 2 \ \text{by} \ 2.$
	
	Solution: This is not a statement, it is a command or step in solving some sort of equation. There is also ambiguity in what x is defined as, there is not enough context.
	
	\item Is $3 \times 6 = 15$?
	
	Solution: This is not a statement because it is a question.
	
	\item The integer $0$ is neither even nor odd.
	
	Solution: This is a statement, but is false because $0$ is even. The definition of an even number is any number that can be written in the form $x = 2n, \, n \in \Z$. Since $0$ is an integer, it will be the result of $2 \times 0 = 0$.
	
	\item If $p\left(x\right)$ is a polynomial of degree $5$, then $p^{'}\left(x\right)$ is a polynomial of degree $4$.
	
	Solution: Ask Oscar, but I believe this is not a statement because $p^{'}\left(x\right)$ is not explicityly defined as the derivative (See 2(g)).
\end{enumerate}

\end{enumerate}


\end{document}


%\vspace{.3in}
%
%\begin{definition}
%This is a command I created: $\WTS$. Use it.
%\end{definition}
%
%\begin{theorem}
%This is a theorem.
%\end{theorem}
%
%\vspace{.3in}
%
%\begin{enumerate}
%\item HERE GOES THE STATEMENT OF PROBLEM 1
%
%\begin{proof}
%HERE GOES THE SOLUTION/PROOF TO PROBLEM 1
%\end{proof}
%
%\vspace{.1in}
%
%\item HERE GOES THE STATEMENT OF PROBLEM 2
%
%\begin{proof}
%HERE GOES THE SOLUTION/PROOF TO PROBLEM 2
%\end{proof}
%
%
%
%
%
%
%\vspace{1in}
%
%
%
%
%
%
%\noindent \textbf{HW reminders.}
%\begin{enumerate}
%\item For Assignments 1 and 2, please submit a pdf file to CANVAS and email me the corresponding tex file by the deadline. Starting in assignment 3, there is no need to submit a tex file to me, just the pdf to CANVAS. 
%
%\item You need to create a group contract by the time HW 2 is due. You do not need to resend this unless changes in your group happen.
%
%\item Please email me weekly your individual weekly report (every week).
%
%\item Please make a copy of this file to write your assignment's solutions. Making changes to it, or to use a different file will create problems.
%\end{enumerate}
%\end{enumerate}
%
%
%
%
%
%
%
%
%\newpage
%
%
%
%
%\begin{itemize}
%\item How to get parentheses, brackets of the right size? Use left and right before them. For example
%\[
%\left(\frac{1}{3}\right)^2 =\frac{1}{9}
%\]
%For curly brackets, it is a little more complicated. You need to use an extra backslash
%\[
%A = \left\{   \frac{1}{3}, \frac{1}{9}    \right\}
%\]
%
%\item The natural numbers, rationals, reals, and complex numbers have special shortcuts: $\N$, $\Q$, $\R$, $\C$.
%
%\item This is OK: $\int_0^1 f(x) \ dx$, but this is better $\ds{\int_0^1 f(x) \ dx}$. You could also do
%\[
%\int_0^1 f(x) \ dx
%\]
%
%\item Similarly,  this is OK: $\lim_{n\to \infty} a_n$, but this is better $\displaystyle{\lim_{n\to \infty} a_n}$. And as an equation, we get
%\[
%\lim_{n\to \infty} a_n
%\]
%
%\item There are more examples of how to do things in the \LaTeX \ workshop file posted on CANVAS.
%\end{itemize}
%
%
%
%
%
%
%
%\end{document}