
\documentclass[12pt]{amsart}
\usepackage{geometry} % see geometry.pdf on how to lay out the page. There's lots.
\usepackage{amsmath}
\newcommand{\innerProd}[2]{\langle #1,#2 \rangle}
\geometry{a4paper} % or letter or a5paper or ... etc
% \geometry{landscape} % rotated page geometry

% See the ``Article customise'' template for come common customisations

\title{Differential Geometry Notes and Problems}
\author{Dillon Allen}

%%% BEGIN DOCUMENT
\begin{document}

\maketitle
\tableofcontents

\section{Day 1: Review}
\subsection{Vectors}

A vector function $\vec{f}$ is defined as 

\[
	\vec{f}\left(t\right) = \langle f_1\left(t\right), f_2\left(t\right), \dots \rangle
\]

Vector functions in $\mathbb{R}^3$ have the inner product defined as 

\[
	\vec{f} \cdot \vec{g} = \sum_{i} f_i g_i
\]

Where $i$ indicates the component of the vector.

The derivative of the inner product can be seen as an analog for the product rule as seen in calculus II. This is defined as

\[
	\frac{d}{dt} \langle \vec{f} \left(t\right) , \vec{g} \left( t \right) \rangle = \langle \frac{d}{dt} \vec{f}\left(t\right), \vec{g}\left(t\right) \rangle + \langle \vec{f}\left(t\right), \frac{d}{dt} \vec{g}\left(t\right) \rangle
\]

\begin{proof}

	Let $\vec{f}\left(t\right)$, $\vec{g}\left(t\right)$ be vector valued functions in $\mathbb{R}^n$. WLOG we will drop the function of t notation for brevity. The inner product in 
	$\mathbb{R}^n$ is defined as 
	\begin{equation*}
		\innerProd{f}{g} = \sum_{i} f_i g_i = f_1g_1 + f_2 g_2 + \ldots
	\end{equation*}
	
	Taking the derivative of both sides, we get
	\begin{equation*}
		\begin{aligned}
			\frac{d}{dt} \innerProd{f}{g} &= \frac{d}{dt} \sum_i f_i g_i \\
										  &= \frac{d}{dt} \left(f_1 g_1 + f_2 g_2 + \ldots \right) \\
										  &= f^{'}_1 g_1 + f_1 g^{'}_1 + f^{'}_2 g_2 + f_2 g^{'}_2 + \dots \\
										  &= \left( f^{'}_1 g_1 + f^{'}_2 g_2 + \ldots \right) + \left(f_1 g^{'}_1 + f_2 g^{'}_2 + \ldots \right) \\
										  &= \innerProd{\frac{d}{dt}\vec{f}}{g} + \innerProd{f}{\frac{d}{dt}\vec{g}} \\
		\end{aligned}
	\end{equation*}
	
	
\end{proof}




\end{document}